\documentclass[11pt]{book}
\newcommand\st{\mbox{s.t.\ }}

\begin{document}

\section{Introduction to Number Theory}
\begin{itemize}
	\item Modular arithmetic
	\item Exponentiation
	\item Representation of integers
	\item Fundamental theorem of arithmetic
	\item Sieving
	\item Primes: Mersenne primes, twin primes, Solinas prime
	\item Prime number theorem
	\item Goldbach's Conjecture
	\item Primality tests: Euclid's algorithm
	\item Congruences
	\item Chinese remainder problems
	\item Discrete log problem
\end{itemize}


Discuss algabraic structures.

\section{Introduction to Group Theory}
\begin{itemize}
	\item Note about Galois and Abel
	\item Introduction to groups
	\item General linear groups
	\item Abelian/Commutative groups
	\item Finite groups (show it's abelian)
	\item Cayley tables (show there is only one finite group of three elements)
	\item Subgroups. (Show that we only need to show closure and inverse properties)
	\item Exponentiation on groups
	\item Generators and Cyclic groups (Klein group: abelian, smallest non cyclic)
	\item Permutation groups
	\item Introduction to rings
	\item Abelian/Commutative rings
	\item Revisiting the DLP for groups
\end{itemize}

Group theory is \emph{the study of symmetry}.


Group $(G, *)$ is a set of elements $g_i$ satisfying the four
conditions below, relative to some binary operation $*$.

If the elements of $G$ satisfy the following four properties, then
$G$ is a group.

\begin{itemize}
\item $\exists e \in G \;\st \forall g \in G : eg = ge = g$. (Identity.)
We often write $e=1$ for multiplicative groups, and $e = 0$ for
additive groups.

\item $\forall x,y,z \in G$ : $(xy)z = x(yz)$. (Associativity.)

\item $\forall x \in G, \exists y \in G \;\st xy = yx = e$. (Inverse.)
We write $y = x^{-1}$ for multiplication, $y = -x$ for addition.

\item $\forall x,y \in G : xy \in G$. (Closure.)
\end{itemize}

If commutation holds ($\forall x, y \in G$, $xy = yx$), we say the
group is Abelian. Non-abelian groups exist and are important. For
example, consider the group of $N \times N$ matrices with real
entries and non-zero determinant. Prove this is a group under
matrix multiplication, and show this group is not commutative.

$H$ is a {\em subgroup} of $G$ if it is a group and its elements
form a subset of those of $G$. The identity of $H$ is the same as
the identity of $G$. Once you've shown the elements of $H$ are
closed (ie, under the binary operation, $b(x,y) \in H$ if $x, y
\in H$), then associativity in $H$ follows from closure in $H$ and
associativity in $G$.

For the application to Fermat's Little Theorem you will need to
know that the set $\{1,x,x^2,\cdots\,x^{n-1}\}$ where $n$ is the
lowest positive integer \st $x^n = 1$, called the {\em cyclic
group}, is indeed a subgroup of any group $G$ containing $x$, as
well as $n$ divides the order of $G$.

\end{document}
